% This is a template for doing homework assignments in LaTeX, cribbed from M. Frenkel (NYU) and A. Hanhart (UW-Madison)

\documentclass{article} % This command is used to set the type of document you are working on such as an article, book, or presenation

\usepackage[margin=1in]{geometry} % This package allows the editing of the page layout. I've set the margins to be 1 inch. 

\usepackage{amsmath, amsfonts}  % The first package allows the use of a large range of mathematical formula, commands, and symbols.  The second gives some useful mathematical fonts.

\usepackage[usenames]{color}

\usepackage{graphicx}  % This package allows the importing of images

\usepackage{datetime}

\usepackage[shortlabels]{enumitem} % This package allows for different types of labels in the enumerate environment 


%This allows us to use the theorem and proof environment 
\usepackage{amsthm}
\theoremstyle{plain}
\newtheorem*{theorem*}{Theorem}
\newtheorem*{lemma*}{Lemma}
\newtheorem{theorem}{Theorem}
\theoremstyle{definition}
\newtheorem*{definition*}{Definition}
\newtheorem*{remark*}{Remark}
\newtheorem*{example*}{Example}

%Custom commands.  
\newcommand{\abs}[1]{\left\lvert #1 \right\rvert} %absolute value command

%Custom symbols
\newcommand{\Rb}{\mathbb{R}}




\begin{document}

\begin{center}
    \Large{
        \textbf{Assignment \#3}

        UW-Madison MATH 421
    }
    
    \vspace{5pt}
        
    \normalsize{
        GEOFF YOERGER

        \usdate
        \formatdate{13}{4}{2021}
    }
    
    \vspace{15pt}
\end{center}

\noindent\fbox{\textbf{Exercise \#1:}} Spivak, Chapter 11, Problem 13

\begin{theorem*}
    $\forall x \in \mathbb{R}^+ \; \colon x + \frac{1}{x} \geq 2$
\end{theorem*}

\begin{proof}
    Let $f(x) = x + \frac{1}{x}$.

    Then $f'(x) = 1 - \frac{1}{x^2}$.

    By the algorithm for finding min values of $f$ on $(0, \infty)$:

    $(1)$: Critical points are where

    \begin{align*}
        0 & = 1 - \frac{1}{x^2} \\
        1 & =  \frac{1}{x^2} \\
        x^2 & = 1 \\
        x & = 1 \\
    \end{align*}

    $(2)$: No points of non-differentiability in $(0, \infty)$.

    $(3)$:

    \begin{align*}
        f(1) & = 2 \\
        \lim_{x \to 0^+} f(x) & = \infty \\
        \lim_{x \to \infty} f(x) & = \infty \\
    \end{align*}

    Thus $f$'s minimum value is $2$.

    Hence $\forall x \in \mathbb{R}^+ \; \colon f(x) \geq 2$

    $\implies \forall x \in \mathbb{R}^+ \; \colon x + \frac{1}{x} \geq 2$.
\end{proof}


\noindent\fbox{\textbf{Exercise \#2:}} Spivak, Chapter 11, Problem 26 

\begin{theorem*}
    Suppose $f$ is a polynomial of degree $n$, with $f \geq 0$. (Hence, $n$ must be even).

    $f + f' + f'' + \cdots + f^{(n)} \geq 0$.
\end{theorem*}

\begin{proof} 
    Let $g = f + f^\prime + \dots + f^{(n)}$.
    
    Because $f$ is a polynomial,

    $\forall n \; \colon f^{(n)}$ is a polynomial,

    and $g$ is a polynomial.

    Hence $g$ is differentiable and continuous everywhere.

    Notice that $g' = f' + f'' + \cdots + f^{(n)} + f^{(n+1)} = f' + f'' + \cdots + f^{(n)}$.

    Hence, $g = f + g'$.

    Because $f$ has an even degree, so does $g$, and

    By a theorem proved in class or hwk, $\exists x \; \colon x$ is a minimum point of $g$ on $\mathbb{R}$.

    Thus $g'(x) = 0$.

    Further, $g(y) = f(y) + g'(y) = f(y)$.

    Because $f(y) \geq 0$, $g(y) \geq 0$, and since $g(y)$ is the minimum value, $\forall x \; \colon g(x) = f(x) + f'(x) + \cdots + f^{(n)}(x) \geq 0$.
\end{proof}

\noindent\fbox{\textbf{Exercise \#3:}} Spivak, Chapter 11, Problem 30 (a) 

\begin{theorem*}
    Suppose $\forall x \; \colon f'(x) > g'(x)$, and $f(a) = g(a)$

    $x > a \implies f(x) > g(x)$
    $x < a \implies f(x) < g(x)$
\end{theorem*}

\begin{proof}
    Let $h = f - g$.

    Notice that $h(a) = f(a) - g(a) = 0$.

    Notice that $\forall x \; \colon f'(x) > g'(x) \implies \forall x \; \colon h'(x) > 0$.

    For $x < a$, consider mean value theorem on $[x,a]$.

    Thus, $\exists y \in (x,a) \; \colon h'(y) = \frac{h(a) - h(x)}{a -x} = \frac{-h(x)}{a-x}$.

    Since $h'(y) > 0$ and $x < a \implies a-x > 0$, then $-h(x) > 0 \implies h(x) < 0 \implies f(x) < g(x)$.
    
    Thus $x < a \implies f(x) < g(x)$.

    For $x > a$, consider mean value theorem on $[a,x]$.

    Thus, $\exists y \in (a,x) \; \colon h'(y) = \frac{h(x) - h(a)}{x - a} = \frac{h(x)}{x-a}$.

    Since $h'(y) > 0$ and $x > a \implies x-a > 0$, then $h(x) > 0 \implies f(x) > g(x)$.

    Thus $x > a \implies f(x) > g(x)$.
\end{proof}


\noindent\fbox{\textbf{Exercise \#4:}} Spivak, Chapter 11, Problem 38

\begin{theorem*}
    If $\frac{a_0}{1} + \frac{a_1}{2} + \cdots + \frac{a_n}{n+1} = 0$,

    Then $\exists x \in (0,1) \; \colon a_0 + a_1 x + \cdots + a_n x^n = 0$.
\end{theorem*}

\begin{proof}
    Let $f(x) \frac{a_0}{1} x + \frac{a_1}{2} x^2 + \cdots + \frac{a_n}{n+1} x^{n+1}$,

    Notice that $f(0) = 0$, and $f(1) = \frac{a_0}{1} + \frac{a_1}{2} + \cdots + \frac{a_n}{n+1} = 0$.

    Notice that $f'(x) = a_0 + a_1 x + \cdots + a_n x^n$.

    By Rolles theorem, $\exists y \in (0,1) \; \colon f'(y) = 0$.

    $\implies \exists y \in (0,1) \; \colon a_0 + a_1 x + \cdots + a_n x^n = 0$.

\end{proof}

\noindent\fbox{\textbf{Exercise \#5:}} Spivak, Chapter 11, Problem 43

\begin{theorem*}
    Suppose $f$ is a function where $\forall x > 0 \; \colon f'(x) = \frac{1}{x}$, and $f(1) = 0$.

    $\forall x,y > 0 \; \colon f(xy) = f(x) + f(y)$
\end{theorem*}

\begin{proof}
    Fix $y > 0$.

    Let $g(x) = f(xy)$

    Then $g'(x) = y * f'(xy) = y * \frac{1}{xy} = \frac{1}{x} = f'(x)$.

    By Corollary 2 of Thm 11-4, $f(x) = g(x) + c$ for some $c$.

    Note that $f(1) = 0 = g(1) + c \implies g(1) = -c$, and $g(1) = f(1 * y) = f(y)$.

    Thus $f(x) = g(x) - c = g(x) - f(y)$

    $\implies g(x) = f(xy) = f(x) + f(y)$.

    Since $y > 0$ was arbitrary,

    $\forall x,y > 0 \; \colon f(xy) = f(x) + f(y)$
\end{proof}


\noindent\fbox{\textbf{Exercise \#6:}} Prove that $ \frac{1}{21}  < \sqrt{101} - 10 < \frac{1}{20}$

\begin{proof}
    Let $f(x) = \sqrt{x}$.

    Consider mean value theorem of $f$ on $(100,101)$.

    Thus $\exists c \; \colon f'(c) = \frac{1}{2 \sqrt{c}} = \frac{\sqrt{101} - \sqrt{100}}{101 - 100} = \sqrt{101} - 10$.

    Thus,
    
    \begin{align*}
        & 100 < x < 101 \\
        \implies & 100 < x < 110.25 \\
        \implies & 10 < \sqrt{x} < 10.5 \\
        \implies & \frac{1}{10} < \frac{1}{\sqrt{x}} < \frac{1}{10.5} \\
        \implies & \frac{1}{20} < \frac{1}{2 \sqrt{x}} < \frac{1}{21} \\
        \implies & \frac{1}{20} < \sqrt{101} - 10 < \frac{1}{21} \\
    \end{align*}
\end{proof}

\noindent\fbox{\textbf{Exercise \#7:}} Spivak, Chapter 11, Problem 64

\begin{theorem*}
    Suppose that $f(0) = 0$, and $f'$ is increasing.

    $g(x) = f(x)/x)$ is increasing on $(0, \infty)$
\end{theorem*}

\begin{proof}
    Let $g(x) = \frac{f(x)}{x}$.

    Thus, $g'(x) = \frac{x f'(x) - f(x)}{x^2}$.

    Fix $N > 0$.

    Consider mean value theorem of $f$ on $(0, N)$.

    Thus, $\exists c \in (0, N) \; \colon f'(c) = \frac{f(N) - f(0)}{N - 0} = \frac{f(N)}{N} \implies N f'(c) = f(N)$.

    Since $c < N$ and $N > 0$, $f'(c) < f'(N) \implies N f'(c) < N f'(N)$.

    Hence, $f(N) < N f'(N) \implies N f'(N) - f(N) > 0$.

    Since $N > 0$, $g'(N) = \frac{N f'(N) - f(N)}{N^2} > 0$.

    Since $N > 0$ was arbitrary, $\forall N > 0 \; \colon g'(N) > 0$.

    By Corollary 3 of Them 11-4, $g(x) = \frac{f(x)}{x}$ is increasing on $(0, \infty)$.
\end{proof}

\end{document}
