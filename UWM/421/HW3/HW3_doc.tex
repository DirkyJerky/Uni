% This is a template for doing homework assignments in LaTeX, cribbed from M. Frenkel (NYU) and A. Hanhart (UW-Madison)

\documentclass{article} % This command is used to set the type of document you are working on such as an article, book, or presenation

\usepackage[margin=1in]{geometry} % This package allows the editing of the page layout. I've set the margins to be 1 inch. 

\usepackage{amsmath, amsfonts}  % The first package allows the use of a large range of mathematical formula, commands, and symbols.  The second gives some useful mathematical fonts.

\usepackage{graphicx}  % This package allows the importing of images

\usepackage{datetime}

%This allows us to use the theorem and proof environment 
\usepackage{amsthm}
\theoremstyle{plain}
\newtheorem*{theorem*}{Theorem}
\newtheorem{theorem}{Theorem}
\theoremstyle{definition}
\newtheorem*{definition*}{Definition}

%Custom commands.  
\newcommand{\abs}[1]{\left\lvert #1 \right\rvert} %absolute value command

%Custom symbols
\newcommand{\Rb}{\mathbb{R}}





\begin{document}

\begin{center}
    \Large{
        \textbf{Assignment \#3}

        UW-Madison MATH 421
    }
    
    \vspace{5pt}
        
    \normalsize{
        GEOFF YOERGER

        \usdate
        \formatdate{16}{2}{2021}
    }
    
    \vspace{15pt}
\end{center}


\noindent\fbox{\textbf{Exercise \#1:}} Prove the following theorem by induction.  



\begin{theorem*} If $n$ is a natural number, then $1^2+2^2 + \dots + n^2 = \frac{n(n+1)(2n+1)}{6}$.

\end{theorem*}

\begin{proof} We argue by induction. 

    Let $P(n) = \sum_{i=1}^{n} i^2 = \frac{n(n+1)(2n+1)}{6}$

    \underline{Base case:} $P(1) = $
    \begin{align*}
        P(1) = & & \sum_{i=1}^{1} n^2 &= \frac{1(1+1)(2(1)+1)}{6} \\
        = & & 1^2 &= \frac{1*2*3}{6} \\
        = & & 1 &= \frac{6}{6} \\
        = & & 1 &= 1 \\
        = & & TRUE & \\
    \end{align*}

    \underline{Induction step:} 
    Let $n \in \mathbb{N}$.  Suppose $P(n)$ is true.
    \begin{align*}
        P(n+1) &= \sum_{i=1}^{n+1} i^2 &= \frac{(n+1)((n+1)+1)(2(n+1)+1)}{6}
               &= \sum_{i=1}^{n} i^2 + (n+1)^2 = \frac{(n+1)(n+2)(2n+3)}{6}
               &=  i^2 + (n+1)^2 = \frac{(n+1)(n+2)(2n+3)}{6}
    \end{align*}

    \underline{Induction step:} 

\end{proof} 

\noindent\fbox{\textbf{Exercise \#2:}} Prove the following theorem by induction.  

\begin{theorem*}[Bernoulli's inequality] Suppose $n$ is a natural number and $x$ is a real number. If $x > -1$, then 
$$
(1+x)^n \geq 1 + nx.
$$
\end{theorem*} 

\begin{proof} We argue by induction. 

\underline{Base case:}

\underline{Induction step:} 

\end{proof} 
 



\noindent \fbox{\textbf{Exercise \#3:}} For this problem we need the following definition: 

\begin{definition*}
An integer $n$ is \emph{divisible} by an integer $k$ if the ratio $n/k$ is an integer. 
\end{definition*}

For example: -3, 0, 3, 6 are all divisible by 3 while 1, 2, 4, 5 are not divisible by 3.  Prove the following: 


\begin{theorem*} Suppose $n$ is an integer. If $n^2$ is divisible by $3$, then $n$ is divisible by $3$. 

\end{theorem*}

 \begin{proof} (Hint: if $n$ is not divisible by $3$, then $n=3k+1$ or $n=3k+2$ for some integer $k$.) \end{proof} 
 
 \noindent \fbox{\textbf{Exercise \#4:}} Prove that $\sqrt{3}$ is irrational.  

 \begin{proof} \end{proof} 

\noindent \fbox{\textbf{Exercise \#5:}} Prove that $\sqrt{2} + \sqrt{3}$ and $\sqrt{2}-\sqrt{3}$ are both irrational.  

 \begin{proof} (Hint: $(a+b)(a-b) = a^2-b^2$) \end{proof} 
 
  \noindent \fbox{\textbf{Exercise \#6:}} Spivak, Chapter 3, Problem 14. 
  
  \begin{proof}[Solution] (justify your answer)
  
  \end{proof} 

 
 
  \noindent \fbox{\textbf{Exercise \#7:}} Spivak, Chapter 3, Problem 23. 


  \begin{proof}[Proof of (a)]  \end{proof} 
  
    \begin{proof}[Proof of (b)]  \end{proof} 
    
     \noindent \fbox{\textbf{Exercise \#8:}} Spivak, Chapter 3, Problem 26. 
 
 
     \begin{proof} \end{proof} 
    
\section{Extra Credit Questions} 

Each extra credit question is worth 1 extra point. \\

\noindent\fbox{\textbf{Exercise E.C.\#1:}} Spivak, Chapter 2, Problem 17

\smallskip

\noindent\fbox{\textbf{Exercise E.C.\#2:}} Spivak, Chapter 3, Problem 16

\smallskip


\noindent\fbox{\textbf{Exercise E.C.\#3:}} Spivak, Chapter 3, Problem 17

\smallskip


\noindent\fbox{\textbf{Exercise E.C.\#4:}} Spivak, Chapter 3, Problem 20 (b)




    
    
\end{document}
