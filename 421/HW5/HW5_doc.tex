% This is a template for doing homework assignments in LaTeX, cribbed from M. Frenkel (NYU) and A. Hanhart (UW-Madison)

\documentclass{article} % This command is used to set the type of document you are working on such as an article, book, or presenation

\usepackage[margin=1in]{geometry} % This package allows the editing of the page layout. I've set the margins to be 1 inch. 

\usepackage{amsmath, amsfonts, amssymb}  % The first package allows the use of a large range of mathematical formula, commands, and symbols.  The second gives some useful mathematical fonts.

\usepackage{graphicx}  % This package allows the importing of images

\usepackage[shortlabels]{enumitem} % This package allows for different types of labels in the enumerate environment 

\usepackage{datetime}

\usepackage{changepage}

%This allows us to use the theorem and proof environment 
\usepackage{amsthm}
\theoremstyle{plain}
\newtheorem*{theorem*}{Theorem}
\newtheorem{theorem}{Theorem}
\theoremstyle{definition}
\newtheorem*{definition*}{Definition}

%Custom commands.  
\newcommand{\abs}[1]{\left\lvert #1 \right\rvert} %absolute value command

%Custom symbols
\newcommand{\Rb}{\mathbb{R}}




\begin{document}

\begin{center}
    \Large{
        \textbf{Assignment \#3}

        UW-Madison MATH 421
    }
    
    \vspace{5pt}
        
    \normalsize{
        GEOFF YOERGER

        \usdate
        \formatdate{2}{3}{2021}
    }
    
    \vspace{15pt}
\end{center}
\vspace{15pt}
        
\noindent\fbox{\textbf{Exercise \#1:}} In this problem we investigate left and right hand limits which are defined as follows. 

\begin{definition*} We write $\lim_{x \rightarrow a^+} f(x) = L$ if for every $\epsilon > 0$ there exists $\delta > 0$ such that: if $x >a$ and $0< \abs{x-a} < \delta$, then $\abs{f(x)-L} < \epsilon$. 

\end{definition*}

\begin{definition*} We write $\lim_{x \rightarrow a^-} f(x) = L$ if for every $\epsilon > 0$ there exists $\delta > 0$ such that: if $x <a$ and $0< \abs{x-a} < \delta$, then $\abs{f(x)-L} < \epsilon$. 

\end{definition*}

Prove the following. 

\begin{theorem*} $\lim_{x \rightarrow a} f(x) = L$ if and only if $\lim_{x \rightarrow a^+} f(x) = L$ and $\lim_{x \rightarrow a^-} f(x) = L$.
\end{theorem*}

\begin{proof} 

    $(\lim_{x \rightarrow a} f(x) = L \implies \lim_{x \rightarrow a^+} f(x) = L \land \lim_{x \rightarrow a^-} f(x) = L)$:

    \begin{adjustwidth}{3em}{0pt}

    Because $\lim_{x \to a} f(x) = L$, $\forall \epsilon > 0 \; \exists \delta > 0 \; \colon 0 < |x-a| < \delta \implies |f(x) - L| < \epsilon$

    If $0 < |x-a| < \delta$, it follows that $x \neq a$.  Thus either $x>a$ or $x<a$. Then,

    whenever $x > a$, $x > a \land 0 < |x-a| < \delta \equiv 0 < |x-a| < \delta \implies |f(x)-L| < \epsilon$.  Thus, $\lim_{x \rightarrow a^+} f(x) = L$.  Since $x \nless a$, $x < a \land 0 < |x-a| < \delta \implies |f(x)-L| < \epsilon$, holds because of the falsity of the implicator.  Thus, $\lim_{x \rightarrow a^-} f(x) = L$.

    Likewise, whenever $x < a$, $x < a \land 0 < |x-a| < \delta \equiv 0 < |x-a| < \delta \implies |f(x)-L| < \epsilon$.  Thus, $\lim_{x \rightarrow a^-} f(x) = L$.  Since $x \ngtr a$, $x > a \land 0 < |x-a| < \delta \implies |f(x)-L| < \epsilon$ holds, because of the falsity of the implicator.  Thus, $\lim_{x \rightarrow a^+} f(x) = L$.

    Thus, $(\lim_{x \rightarrow a} f(x) = L \implies \lim_{x \rightarrow a^+} f(x) = L \land \lim_{x \rightarrow a^-} f(x) = L)$.

    \end{adjustwidth}

    $(\lim_{x \rightarrow a^+} f(x) = L \land \lim_{x \rightarrow a^-} f(x) = L \implies \lim_{x \rightarrow a} f(x) = L)$:

    \begin{adjustwidth}{3em}{0pt}

    Because $\lim_{x \to a^+} f(x) = L$, $\forall \epsilon > 0 \; \exists \delta > 0 \; \colon x>a \land 0 < |x-a| < \delta \implies |f(x) - L| < \epsilon$.

    Because $\lim_{x \to a^-} f(x) = L$, $\forall \epsilon > 0 \; \exists \delta > 0 \; \colon x<a \land 0 < |x-a| < \delta \implies |f(x) - L| < \epsilon$.

    If $0 < |x-a| < \delta$, it follows that $x \neq a$.  Thus either $x>a$ or $x<a$. Then,

        whenever $x > a$, $x > a \land 0 < |x-a| < \delta \implies |f(x)-L| < \epsilon \equiv 0 < |x-a| < \delta \implies |f(x)-L| < \epsilon$.  Thus, $\lim_{x \rightarrow a} f(x) = L$.

        Likewise, whenever $x < a$, $x < a \land 0 < |x-a| < \delta \implies |f(x)-L| < \epsilon \equiv 0 < |x-a| < \delta \implies |f(x)-L| < \epsilon$.  Thus, $\lim_{x \rightarrow a} f(x) = L$.

    Thus, $(\lim_{x \rightarrow a} f(x) = L$.

    \end{adjustwidth}


\end{proof} 


\noindent\fbox{\textbf{Exercise \#2:}} Consider the function 
$$f(x) = \begin{cases} 0 & \text{if $x$ is irrational} \\
1 & \text{if $x$ is rational.}
\end{cases}
$$
Using the $\epsilon/\delta$ definition, prove that $\lim_{x \rightarrow a} f(x)$ does not exist for every real number $a$ (and hence $f$ is discontinuous at every real number). 

\begin{proof}
    Suppose for a contradiction that $\exists a \; \colon \lim_{x \to a} f(x) = L$. 
    
    Hence $\forall \epsilon > 0 \; \exists \delta > 0 \; \colon 0 < |x-a| < \delta \implies |f(x) - L| < \epsilon$.
    
    Notice that $\forall x \; \colon f(x) \in \{0, 1\}$.

    Let $\epsilon = min\{|L|, |L-1|\}$.

    When $f(x) = 1$, $|f(x)-L| = |1-L| = |L-1| \nless \epsilon = |L-1|$.

    When $f(x) = 0$, $|f(x)-L| = |0-L| = |L| \nless \epsilon = |L|$.

    Thus, $\exists \epsilon$ where $\forall \delta \; \: 0 < |x-a| < \delta \nrightarrow |f(x)-L| < \epsilon$.

    Hence, $\exists a \; \colon \lim_{x \to a} f(x) = L$ is a contradiction, and $\forall a \; \colon \lim_{x \to a} f(x) \text{ DNE}$.

\end{proof} 



\noindent\fbox{\textbf{Exercise \#3:}} Prove the following
\begin{enumerate}[(a)]
\item The function $f(x) = x$ is continuous.
\item If $n$ is a natural number, then the function $f_n(x) = x^n$ is continuous. 
\item If $g$ is a polynomial, then $g$ is continuous on $\Rb$.
\item If $h$ is a rational function, then $h$ is continuous at every point in its domain. 
\end{enumerate}

(Hint: b,c,d should follow quickly from results in class) 

\begin{proof}[Proof of (a)] 
    "$f(x) = x$ is continuous" means $\forall a \; \forall \epsilon > 0 \; \exists \delta > 0\; \colon 0 < |x-a| < \delta \implies |f(x) - f(a)| < \epsilon$.

    Fix $a$. Fix $\epsilon$. Let $\delta = \epsilon$.  If $0 < |x-a| < \delta$, then $|f(x) - f(a)| = |x-a| < \delta = \epsilon$.

    Since $\epsilon$ was arbitrary, $\forall \epsilon > 0 \; \exists \delta > 0 \; \colon 0 < |x-a| < \delta \implies |f(x) - f(a)| < \epsilon$.

    Since $a$ was arbitrary, $\forall a \; \forall \epsilon > 0 \; \exists \delta > 0 \; \colon 0 < |x-a| < \delta \implies |f(x) - f(a)| < \epsilon$.

\end{proof} 

\begin{proof}[Proof of (b)] 
    Because $f(x) = x$ is continuous, and using the theorem proved in class, it follows from induction that $f(x) = x^2 = x \cdot x$ is continuous, $f(x) = x^3 = x^2 \cdot x$ is continuous, $\cdots$.  Hence, $\forall n \in \mathbb{N} \; \colon f(x) = x^n$ is continuous. 
\end{proof} 

\begin{proof}[Proof of (c)] 
    By definition, $g(x) = a_0 + a_1 \cdot x + a_2 \cdot x^2 + \cdots + a_n \cdot x^n$ for some numbers $a_0, a_1, \cdots, a_n$.

    Because the functions $f(x) = a_0$, $f(x) = a_1$, $\cdots$, $f(x) = a_n$ are continuous, and the functions $f(x) = x$, $f(x) = x^2$, $\cdots$, $f(x) = x^n$ are continuous, it follows from the theorem proved in class that the functions $f(x) = a_0$, $f(x) = a_1 \cdot x$, $f(x) = a_2 \cdot x^2$, $\cdots$, $f(x) = a_n \cdot x^n$ are continuous.

    It follows from induction using the theorem proved in class that the functions $f(x) = a_0 + a_1 \cdot x$, $f(x) = a_0 + a_1 \cdot x + a_2 \cdot x^2 = (a_0 + a_1 \cdot x) + a_2 \cdot x^2$, $\cdots$, $f(x) = (a_0 + a_1 \cdot x + \cdots + a_{n-1} \cdot x^{n-1}) + a_n \cdot x^n = a_0 + a_1 \cdot x + \cdots + a_n \cdot x^n$ are continuous.
\end{proof} 

\begin{proof}[Proof of (d)] 
    Notice that, by definintion, the domain of a rational function $f(x) = \frac{g(x)}{h(x)}, g,h \text { are polynomials }$ is wherever $h(x) \neq 0$.

    It follows from the theorem proved in the class, because $g(x)$ and $h(x)$ are polynomials that are continuous, that $f(x)$ is continuous in its domain.
\end{proof} 

\noindent\fbox{\textbf{Exercise \#4:}} Spivak, Chapter 6, Problem 3 (a). 

\begin{theorem*} Given a function $f$ where $\forall x \; \colon |f(x)| \leq |x|$, $f$ is continuous at $0$.
\end{theorem*}

\begin{proof} 
    Notice that when $x=0$, $|f(0)| \leq |0| \implies f(0) = 0$
    
    $\lim_{x \to 0} f(x) = 0$ means $\forall \epsilon \; \exists \delta \; \colon |x| < \delta \implies |f(x)| < \epsilon$.

    Fix $\epsilon$. Let $\delta = \epsilon$.

    If $|x| < \delta$, then $|f(x)| \leq |x| < \delta = \epsilon$.

    Since $\epsilon$ was arbitrary,  $\lim_{x \to 0} f(x) = 0$.
\end{proof} 

    
    
\end{document}
