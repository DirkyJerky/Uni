% This is a template for doing homework assignments in LaTeX, cribbed from M. Frenkel (NYU) and A. Hanhart (UW-Madison)

\documentclass{article} % This command is used to set the type of document you are working on such as an article, book, or presenation

\usepackage[margin=1in]{geometry} % This package allows the editing of the page layout. I've set the margins to be 1inch. 
\usepackage{datetime}

\usepackage{amsmath, amsfonts}  % The first package allows the use of a large range of mathematical formula, commands, and symbols.  The second gives some useful mathematical fonts.

\usepackage{graphicx}  % This package allows the importing of images

%Custom commands 
\newcommand{\der}[2]{\frac{\mathrm{d} #1}{\mathrm{d} #2}}

%Custom symbols
\newcommand{\linediv}{\noindent\rule{6.5in}{2pt}}
\newcommand{\linedivm}{\noindent\rule{6.5in}{1pt}}


\begin{document}

\begin{center}
    \Large{
        \textbf{Assignment \#1}

        UW-Madison MATH 421
    }
    
    \vspace{5pt}
        
    \normalsize{
        GEOFF YOERGER

        \usdate
        \formatdate{2}{2}{2021}
    }
    
    \vspace{15pt}
\end{center}
    
\subsection*{Exercise One:} 
    
a)

\linediv

\begin{verbatim}
If $f(x) = x^n$,  then 
$$
f^\prime(x) = n x^{n-1}.
$$
\end{verbatim}

\linedivm

If $f(x) = x^n$,  then 
$$
f^\prime(x) = n x^{n-1}.
$$

\linediv
 
\noindent b)

\linediv

\begin{verbatim}
If $n \neq -1$,  then 
$$
\int x^{n} dx = \frac{1}{n+1} x^{n+1} + C.
$$
\end{verbatim}

\linedivm

If $n \neq -1$,  then 
$$
\int x^{n} dx = \frac{1}{n+1} x^{n+1} + C.
$$

\linediv

\noindent c)

\linediv

\begin{verbatim}
The derivative of a function $f$ at $x=a$ is
$$
f'(a) = \lim_{h \to 0} \frac{f(a+h) - f(a)}{h}.
$$
\end{verbatim}

\linedivm

The derivative of a function $f$ at $x=a$ is
$$
f'(a) = \lim_{h \to 0} \frac{f(a+h) - f(a)}{h}.
$$

\linediv

    
\subsection*{Exercises Two:}     
    
\noindent a)

\linediv

\begin{verbatim}
The number $e$ is defined by
$$
e = \lim_{n \to \inf} \left( 1 + \frac{1}{n} \right) ^ {n}.
$$
\end{verbatim}

\linedivm

The number $e$ is defined by
$$
e = \lim_{n \to \infty} \left( 1 + \frac{1}{n} \right) ^ {n}.
$$

\linediv

\noindent b)

\linediv

\begin{verbatim}
If $f$ is a continuous function, then
$$
\frac{d}{dx} \left[ \int_{a}^{x} f(t) dt \right] = f(x).
$$
\end{verbatim}

\linedivm

If $f$ is a continuous function, then
$$
\frac{d}{dx} \left[ \int_{a}^{x} f(t) dt \right] = f(x).
$$

\linediv
 
\subsection*{Exercise Three:}

\linediv

\begin{verbatim}
\begin{center}
    \begin{tabular}{|rl|cc|}
    \hline
    First Name & Last Name & Ice Cream Flavor & Number of Scoops \\
    \hline
    Alexa & Leal & Vanilla & $4$ \\
    Julia & Maschi & Chocolate & $2$ \\
    Johnny & Tran & Strawberry & $18$ \\
    Geoff & Yoerger & Chocolate Malt & $\infty$ \\
    \hline
    \end{tabular}
\end{center}
\end{verbatim}

\linedivm

\begin{center}
    \begin{tabular}{|rl|cc|}
    \hline
    First Name & Last Name & Ice Cream Flavor & Number of Scoops \\
    \hline
    Alexa & Leal & Vanilla & $4$ \\
    Julia & Maschi & Chocolate & $2$ \\
    Johnny & Tran & Strawberry & $18$ \\
    Geoff & Yoerger & Chocolate Malt & $\infty$ \\
    \hline
    \end{tabular}
\end{center}

\linediv

\subsection*{Exercise Four:}

\linediv

\begin{verbatim}
$$
\det
\begin{pmatrix}
    a & b \\
    c & d \\
\end{pmatrix}
= ad - bc.
$$
\end{verbatim}

\linedivm

$$
\det
\begin{pmatrix}
    a & b \\
    c & d \\
\end{pmatrix}
= ad - bc.
$$

\linediv

\subsection*{Exercise Five:}

\noindent 1)

\linediv

\begin{verbatim}
\newcommand{\Rb}{\mathbb{R}}
\newcommand{\Nb}{\mathbb{N}}
\newcommand{\Zb}{\mathbb{Z}}
\newcommand{\Qb}{\mathbb{Q}}
\newcommand{\Cb}{\mathbb{C}}

$$
\Nb \subset \Zb \subset \Qb \subset \Rb \subset \Cb
$$
\end{verbatim}

\linedivm

\newcommand{\Rb}{\mathbb{R}}
\newcommand{\Nb}{\mathbb{N}}
\newcommand{\Zb}{\mathbb{Z}}
\newcommand{\Qb}{\mathbb{Q}}
\newcommand{\Cb}{\mathbb{C}}

$$
\Nb \subset \Zb \subset \Qb \subset \Rb \subset \Cb
$$

\linediv

\noindent 2)

\linediv

\begin{verbatim}
\newcommand{\pder}[2]{\frac{\partial #1}{\partial #2}}

If $z = x^2 + xy + y^2$, then
$$
\pder{z}{x} = 2x + y
$$
\end{verbatim}

\linedivm

\newcommand{\pder}[2]{\frac{\partial #1}{\partial #2}}

If $z = x^2 + xy + y^2$, then
$$
\pder{z}{x} = 2x + y
$$

\linediv
 
\subsection*{Exercise Six:}
    
\linediv

\begin{verbatim}
If $f(x) = x^2$, then
\begin{align*}
    f'(a) & = \lim_{h \to 0} \frac{f(a+h) - f(a)}{h} \\
    & = \lim_{h \to 0} \frac{(a+h)^2 - a^2}{h} \\
    & = \lim_{h \to 0} \frac{a^2 + 2ah + h^2 - a^2}{h} \\
    & = \lim_{h \to 0} 2a + h = 2a. \\
\end{align*}
\end{verbatim}

\linedivm

If $f(x) = x^2$, then
\begin{align*}
    f'(a) & = \lim_{h \to 0} \frac{f(a+h) - f(a)}{h} \\
    & = \lim_{h \to 0} \frac{(a+h)^2 - a^2}{h} \\
    & = \lim_{h \to 0} \frac{a^2 + 2ah + h^2 - a^2}{h} \\
    & = \lim_{h \to 0} 2a + h = 2a. \\
\end{align*}

\linediv

\end{document}
